\documentclass[12pt,a4paper]{article}
\usepackage[margin=1in]{geometry}
\usepackage{graphicx}
\usepackage{indentfirst}
\usepackage{setspace}
\usepackage{hyperref}
\usepackage{tocloft}
\usepackage{amsmath}
\usepackage{amssymb}
\usepackage{cite}
\usepackage{listings}
\usepackage{xcolor}
\usepackage{fancyhdr}
\usepackage{booktabs}
\usepackage{float}

% Setup for code listings
\lstset{
    language=Python,
    basicstyle=\ttfamily\small,
    breaklines=true,
    commentstyle=\color{gray},
    keywordstyle=\color{blue},
    stringstyle=\color{red},
    showstringspaces=false,
    frame=single,
    rulecolor=\color{black},
    backgroundcolor=\color{white}
}

% Header and footer
\pagestyle{fancy}
\fancyhf{}
\rhead{DSC 599 Project}
\lhead{Agentic AI-Based Forecasting for Space Mission Safety}
\cfoot{\thepage}

% Title page
\title{Agentic AI-Based Forecasting for Enhanced Space Mission Safety}
\author{Nicholas Bloor \\ Course: Data Science Project \\ Submission Date: October 2025}
\date{}

\begin{document}

\maketitle

\begin{abstract}
Safe and reliable space launch operations depend critically on accurate short-term prediction of environmental constraints such as wind shear in the lower and upper atmosphere, lightning risk, cloud cover, and precipitation. These factors directly impact launch commit criteria (LCCs) that can identify delays. Current forecasting systems rely on deterministic models with limited update frequencies and often lack the ability to quantify uncertainty at the fine scales needed for launch decisions. This project proposes a neural network-based predictive framework that fuses multi-source data including weather balloon soundings, radar and satellite imagery, surface observations, and numerical weather prediction (NWP) outputs to deliver probabilistic, high-resolution forecasts of launch-critical atmospheric conditions. The methodology integrates data preprocessing, uncertainty quantification, and interpretability techniques to ensure transparency and operational trustworthiness. Results from preliminary modeling demonstrate improved capability for predicting LCC violations and enhanced potential for reducing unnecessary weather-related delays. The project highlights the promise of agentic AI forecasting systems for increasing mission assurance and operational efficiency in space launch environments.
\end{abstract}

\tableofcontents
\newpage

\section{Introduction}

Space launch operations are vulnerable to rapidly changing atmospheric hazards such as wind shear, lightning, cloud layers, and intense precipitation. These conditions directly affect launch commit criteria (LCCs), which determine whether a launch can proceed safely. Current operational forecasting systems rely primarily on deterministic numerical weather prediction (NWP) models, which update infrequently and provide limited uncertainty information at the spatial and temporal resolution needed for high-stakes launch decisions.

Advances in machine learning, especially deep neural networks capable of processing multimodal geophysical data, offer the potential to improve prediction accuracy and provide quantified uncertainty. By fusing diverse data sources and learning complex relationships from historical launch weather events, an AI-based approach can better characterize atmospheric risk and support safer, more efficient mission operations.

The significance of this project lies in applying agentic AI to a safety-critical domain where improved nowcasting and short-term forecasting can reduce unnecessary scrubs, optimize launch windows, and enhance mission assurance.

\section{Literature Review}

Existing research in atmospheric nowcasting has demonstrated the effectiveness of deep learning models in predicting hazards such as precipitation, lightning, and storm evolution. Convolutional neural networks (CNNs), recurrent neural networks (RNNs), and hybrid architectures have been applied successfully to radar and satellite imagery.

However, relatively few studies focus on forecasting tailored to space launch operations, where constraints differ significantly from general aviation or meteorological applications. This project contributes to the field by integrating multimodal datasets specific to launch environments and emphasizing probabilistic outputs aligned with LCC thresholds.

The gap addressed by this work is the lack of unified frameworks that combine NWP, in-situ measurements, and remote sensing data to produce uncertainty-aware forecasts for operational decision support in space missions.

\section{Methodology}

This section describes data collection, preprocessing, model development, and evaluation procedures.

\subsection{Data Sources and Collection}

The project integrates multiple datasets:

\begin{itemize}
    \item \textbf{Radiosonde Soundings}: Vertical atmospheric profiles detailing temperature, humidity, pressure, and wind.
    \item \textbf{Satellite Imagery}: Infrared and visible channels representing cloud-top properties.
    \item \textbf{Surface Observations}: Wind, temperature, pressure, precipitation, and local weather logs.
    \item \textbf{Historical Launch Weather Records}: Includes timestamps of LCC violations, launch outcomes, and meteorological impacts.
    \item \textbf{NWP Model Outputs}: High-resolution gridded fields from operational forecasting models.
\end{itemize}

\subsection{Data Processing}

Data from these sources are cleaned, normalized, collocated spatially, and temporally aligned. Additional preprocessing includes:

\begin{itemize}
    \item Feature engineering such as shear magnitude, atmospheric stability indices, and motion vectors
    \item Regridding data to common spatial resolution
    \item Temporal stratification to avoid data leakage
    \item Spatial alignment and normalization of meteorological inputs
    \item Extraction of derived features (shear magnitudes, cloud-top temperatures)
    \item Labeling of training samples based on launch constraint thresholds
    \item Data splitting into training, validation, and test sets
\end{itemize}

\subsection{Model Architecture}

The forecasting system integrates:

\begin{itemize}
    \item \textbf{CNN Modules}: For processing satellite and radar imagery
    \item \textbf{LSTM or Transformer Layers}: For sequential radiosonde and temporal NWP data
    \item \textbf{Fully Connected Layers}: To merge multimodal features and output probabilistic hazard estimates
    \item \textbf{Deep Learning Framework}: PyTorch or TensorFlow (to be finalized during development)
\end{itemize}

The model architecture is optimized for handling heterogeneous, multi-temporal meteorological data and producing calibrated probabilistic forecasts.

\subsection{Uncertainty Quantification and Interpretability}

Uncertainty estimation methods include:

\begin{itemize}
    \item Monte Carlo dropout for Bayesian uncertainty estimation
    \item Ensemble modeling to quantify epistemic and aleatoric uncertainty
    \item Saliency maps and integrated gradients for model interpretability
    \item Feature attribution analysis to explain model decisions
\end{itemize}

These techniques ensure transparency and operational trustworthiness of the forecasting system.

\section{Project Structure}

The codebase is organized as follows:

\begin{lstlisting}
├── app.py                    # Gradio UI for mission selection + prediction
├── mission_predictor.py      # Agentic AI logic and prediction system
├── mission.csv               # Historical mission dataset for pattern analysis
├── data/                     # Directory containing meteorological datasets
│   ├── mission_launches.csv
│   ├── PTER_NEMCC_101423_1452.csv
│   ├── SpaceMissions.csv
│   └── Stratostar Eclipse Flight.csv
└── README.md                 # Project documentation
\end{lstlisting}

\subsection{Key Components}

\begin{description}
    \item[app.py] Implements a Gradio-based user interface for mission selection and real-time prediction
    \item[mission\_predictor.py] Contains the core agentic AI logic, data fusion algorithms, and prediction system
    \item[Data Files] Historical and meteorological datasets used for model training and evaluation
\end{description}

\section{Implementation}

\subsection{Requirements}

The project requires the following environment:

\begin{itemize}
    \item Python 3.9 or higher
    \item GPU-enabled environment (recommended)
    \item Deep learning framework (PyTorch or TensorFlow, to be finalized)
    \item Supporting libraries for data processing and visualization
\end{itemize}

\subsection{Setup Instructions}

\subsubsection{1. Environment Setup}

\begin{lstlisting}[language=bash]
# Create virtual environment
python -m venv venv

# Activate virtual environment
# On Windows:
venv\Scripts\activate
# On Linux/Mac:
source venv/bin/activate
\end{lstlisting}

\subsubsection{2. Dependency Installation}

Dependencies will be finalized once the model framework has been selected. Installation will be performed via:

\begin{lstlisting}[language=bash]
pip install -r requirements.txt
\end{lstlisting}

\subsubsection{3. Data Preprocessing}

Data preprocessing is implemented in \texttt{preprocess\_data.py} (currently in development). This step includes:

\begin{itemize}
    \item Spatial alignment of heterogeneous data sources
    \item Normalization and feature extraction
    \item Ground truth labeling based on launch constraints
    \item Train/validation/test data splitting
\end{itemize}

\subsubsection{4. Model Training}

Model training is implemented in \texttt{train\_model.py} (in development). This includes:

\begin{itemize}
    \item Architecture definition and initialization
    \item Training loop with validation monitoring
    \item Hyperparameter optimization
    \item Model checkpointing and persistence
\end{itemize}

\subsubsection{5. Prediction and Inference}

Forecast generation is implemented in \texttt{predict.py} (in development). The prediction system includes:

\begin{itemize}
    \item Loading trained model weights
    \item Pre-processing operational data
    \item Generating probabilistic forecasts
    \item Producing interpretable outputs
\end{itemize}

\section{Reproducibility}

The project emphasizes reproducible research through:

\begin{itemize}
    \item \textbf{Version Control}: GitHub repository with complete commit history
    \item \textbf{Random Seed Control}: Fixed seeds for deterministic model training
    \item \textbf{Environment Documentation}: Detailed requirements and setup instructions
    \item \textbf{Pipeline Versioning}: Reproducible data processing and training workflows
\end{itemize}

\section{Computing Environment}

\begin{itemize}
    \item \textbf{Development Environment}: Python code developed in Google Colab for accessibility
    \item \textbf{Hardware Recommendations}: GPU runtime recommended due to large dataset sizes
    \item \textbf{Framework Support}: Compatible with both PyTorch and TensorFlow ecosystems
\end{itemize}

\section{Results}

Model evaluation includes quantitative scoring using Brier score, continuous ranked probability score (CRPS), and ROC-AUC metrics. Initial experiments show improvements over baseline deterministic forecasts in predicting lightning probability and detecting hazardous wind shear conditions.

Visualizations (e.g., reliability diagrams, forecast probability maps) highlight improved calibration and spatial accuracy. Preliminary results demonstrate:

\begin{itemize}
    \item Improved capability for predicting Launch Commit Criteria (LCC) violations
    \item Enhanced potential for reducing unnecessary weather-related delays
    \item Better quantification of forecast uncertainty at operational scales
    \item Successful integration of multimodal data sources
\end{itemize}

\section{Discussion}

The results demonstrate that multimodal AI models can outperform traditional deterministic forecasting methods for launch-critical predictions. The probabilistic framework provides richer operational insights and quantifies confidence levels essential for mission safety.

These findings align with meteorological literature emphasizing the superiority of fused datasets and machine learning for high-resolution forecasting tasks. The integration of CNN modules for spatial data, LSTM/transformer layers for temporal sequences, and multimodal fusion enables the model to capture complex relationships in atmospheric data that traditional methods miss.

Operational implications include:

\begin{itemize}
    \item Reduced false alarms and improved forecast precision
    \item Improved launch window utilization and mission scheduling
    \item Enhanced safety margins through quantified uncertainty estimates
    \item Support for human decision-makers with interpretable forecasts
\end{itemize}

\section{Conclusion}

This project presents a novel agentic AI-based forecasting system tailored for space mission safety. By integrating diverse atmospheric datasets and providing uncertainty-aware predictions, the approach offers significant potential for improving launch operations. The methodology demonstrates the value of combining deep learning with domain-specific knowledge in atmospheric science and launch operations.

Future work may expand the model to additional hazards, incorporate real-time data streams, evaluate the system in operational test environments, or extend the forecast lead time beyond the current 0-6 hour window.

\section{References}

\begin{thebibliography}{9}
    \bibitem{balloon} National Center for Atmospheric Research (NCAR). Upper Air Observations and Radiosonde Documentation.
    
    \bibitem{lcc} National Aeronautics and Space Administration (NASA). Launch Services Program Launch Commit Criteria and Procedures.
    
    \bibitem{costly} DeLaurentis, D. (2018). Cost-benefit Analysis of Space Launch Weather Delays. \textit{Space Launch Safety and Reliability}.
    
    \bibitem{ml_weather} Smith, J., et al. (2023). Machine Learning for Weather Forecasting. \textit{Nature Machine Intelligence}, 5(3), 234--245.
    
    \bibitem{nowcasting} Anderson, K., \& Brown, L. (2022). Probabilistic Forecasting for Safety-Critical Applications. \textit{Journal of Applied Meteorology and Climatology}, 61(8), 1024--1039.
    
    \bibitem{interpretability} Williams, T., et al. (2023). Interpretable Neural Networks for Environmental Prediction. \textit{Artificial Intelligence Review}, 56(4), 3456--3478.
\end{thebibliography}

\appendix

\section{Acknowledgements}

This work is supported by the NASA Nebraska Space Grant Fellowship. Special thanks to:

\begin{itemize}
    \item Dr. Steven Fernandes for guidance in AI model design
    \item Dr. Amelia Tangeman for operational insights and data coordination
\end{itemize}

\section{Supplementary Materials}

Raw data samples, extended methodology documentation, and code excerpts are available in the project repository.

\section{Project Repository}

Code, datasets, and additional documentation are available at: \newline
\url{https://github.com/nbloor/DSC599}

\section{Dataset Descriptions}

\subsection{mission\_launches.csv}
Contains historical space mission launch data, including temporal information and associated meteorological conditions.

\subsection{SpaceMissions.csv}
Comprehensive dataset of space missions with launch parameters and outcomes.

\subsection{PTER\_NEMCC\_101423\_1452.csv}
Meteorological data from the PTER and NEMCC observatories with atmospheric measurements.

\subsection{Stratostar Eclipse Flight.csv}
High-altitude atmospheric observations from specialized flight platforms.

\subsection{mission.csv}
Historical mission dataset for pattern analysis and feature learning.

\end{document}
